\documentclass[10pt,notes,compress,usetitleprogressbar,aspectratio=1610]{beamer}

\usepackage[spanish, es-tabla]{babel}
\usepackage[T1]{fontenc}
\usepackage[utf8x]{inputenc}
\usepackage{tikztools}
\usepackage{pgfplots}
\usepackage{chronology}
\usepackage{tikz-3dplot}

\definecolor{palette1}{HTML}{1B9E77}
\definecolor{palette2}{HTML}{D95F02}
\definecolor{palette3}{HTML}{7570B3}
\definecolor{palette4}{HTML}{E7298A}
\definecolor{palette5}{HTML}{66A61E}
\definecolor{palette6}{HTML}{E6AB02}
\definecolor{palette7}{HTML}{A6761D}
\definecolor{palette8}{HTML}{666666}

\usetheme{metropolis}

\metroset{
	color/theme = eps,
	fonttheme = palatino,
	titleformat = allsmallcaps,
	titleformat subtitle = regular,
	numbering = counter,
	progressbar = frametitle,
	titleformat plain = regular,
	logo = eps,
	section as frame subtitle
}

\renewcommand{\event}[3][e]{
	\pgfmathsetlength\xstop{(#2-\theyearstart)*\unit}%
	\draw[fill=black,draw=none,opacity=0.5]%
		(\xstop, 0) circle (.2\unit)%
		node[opacity=1,rotate=45,right=0\unit, yshift = 0.3cm] {#3};
}

% For input of gnuplot files. Automatically resizes the image based on linewidth.
\newcommand{\inputgnuplot}[2][1]{
	\vspace{-2ex}
	\resizebox{#1\linewidth}{!}{
		\IfFileExists{#2.tex}{
			\fontsize{7pt}{7pt}\fontfamily{ppl}\selectfont
			\input{#2.tex}
			\vspace{-15pt} % Adjust for font size
		}{
			\PackageWarning{epstfg}{File `#2.tex' not found.}
			\fbox{Gnuplot not found.\begin{minipage}{4.7in}\hfill\vspace{3in}\end{minipage}}
		}
	}
	\vspace{-3ex}
}


\title{Monitorizaci\'on, captura y almacenamiento inteligente de tr\'afico de red a 40 Gbps}
\subtitle{Trabajo Fin de Grado | Doble Grado en Ingenier\'ia Inform\'atica - Matem\'aticas}
\author{Guillermo Juli\'an Moreno}
\date{Junio 2016}

\institute{HPCN \\ Dept. Tecnología Electr\'onica \\ y de las comunicaciones}

% \setbeameroption{hide notes}
% \setbeameroption{show notes}
% \setbeameroption{show only notes}

\begin{document}

% Aclarar que se usa principalmente Intel y Mellanox
\maketitle \note{}

\begin{frame}{Tabla de contenidos}
  \setbeamertemplate{section in toc}[sections numbered]
  \tableofcontents[hideallsubsections]
  \note{Empty}
\end{frame}

\section{Introducción y motivación} \note{En primer lugar, vamos a entender en qué contexto nos estamos moviendo, que es el de las redes Ethernet y su monitorización.}

\begin{frame}
\frametitle{Redes Ethernet}

\begin{figure}[t]
\centering
\begin{chronology}{1999}{2016}{0.8\textwidth}
\event{2002}{\footnotesize 1ª versión 10 GbE}
\event{2006}{\footnotesize Estándar 10 GbE}
\event{2010}{\footnotesize Estándar 40/100 GbE}
\event{2014}{\footnotesize Intel XL710 40 GbE}
\event{2011}{\footnotesize Mellanox ConnectX-3 40 GbE}
\end{chronology}
\caption{Cronología de redes Ethernet de alta velocidad}
\end{figure}

Cada vez más CPDs aumentan la velocidad de sus redes Ethernet: las soluciones a 40~Gbps ya empiezan a desplegarse comercialmente.

Las \textbf{necesidades de monitorización} persisten: detección de problemas y anomalías, diagnósticos de red, detección de intrusiones... \note{A lo largo de los años, las redes Ethernet han ido evolucionando con mayor velocidad cada vez. Estas soluciones son especialmente útiles en CPDs, donde ya se están empezando a desplegar las redes de 40 Gbps, y donde Intel y Mellanox son los actores más potentes.}
\end{frame}

\begin{frame}{Monitorización de redes}

\begin{itemize}
\item Por un lado, se puede analizar el tráfico según llega al sistema para detectar problemas, sacando también estadísticas agregadas. \note[item]{Ejemplos de agregadas: estadísticas por cada conexión TCP, o tasa de tráfico recibido según las redes.}
\item También se puede guardar el tráfico y analizarlo con herramientas más complejas, o estudiar en detalle intervalos en los que hubo problemas.
\end{itemize}

En los dos casos necesitamos herramientas que permitan recoger los paquetes recibidos y nos permitan saber cuándo llegaron y en qué orden. \note[item]{De hecho, si nos llegan paquetes desordenados puede ser un problema de la red.}

\end{frame}

\begin{frame}{Estado del arte}

\note[item]{En un primer momento podríamos pensar en usar directamente la captura del sistema, pero la pila de red no está preparada para hacer captura a alto rendimiento: es genérica y hace demasiadas cosas que no nos hacen falta si sólo queremos guardar los paquetes. Necesitamos herramientas especializadas, y preferiblemente usando hardware estándar.

Hemos revisado algunas de esas soluciones.}

La pila de red del sistema no es suficiente para obtener alto rendimiento: hacen falta soluciones específicas.

\begin{itemize}
\item \textbf{PACKET\_MMAP}: Acceso directo a los anillos de recepción del sistema, integrado en el kernel. \note[item]{Tenemos, por un lado, una ya integrada dentro del sistema que nos da un acceso más directo a los paquetes, pero que sigue siendo demasiado lenta.}
\item \textbf{PF\_RINC ZC}: Sistema comercial que evita la pila de red, orientado a virtualización. Sólo captura a 10 Gbps y no tiene marcas de tiempo. \note[item]{Hay otras soluciones más eficientes que se basan en saltar toda la pila de red del sistema y acceder directamente a las estructuras de la tarjeta. El problema es que o bien no dan soporte para tarjetas de 40 Gbps, o no permiten marcar todos los paquetes que llegan de manera precisa, lo que los hace menos apropiados para análisis de red.}
\item \textbf{Netmap}: Acceso directo a los anillos de la tarjeta sin usar la pila de red, con múltiples anillos y colas para mejorar rendimiento. Sólo funciona a 40 Gbps con tarjetas Chelsio.
\item \textbf{Intel DPDK}: Soporte para múltiples tarjetas, incluyendo las de Mellanox e Intel de 40 GbE. Expone APIs en espacio de usuario para trabajar con los paquetes recibidos.
\item \textbf{HPCAP}: Tesis doctoral de Víctor Moreno (HPCN, Escuela Politécnica Superior), un \textit{driver} Linux para captura y almacenamiento de tráfico a 10 Gbps en tarjetas Intel, con uso bajo de recursos. \note[item]{Un hilo por cada interfaz de recepción.}
\end{itemize}
\end{frame}

\begin{frame}{Arquitectura previa HPCAP}

\note{La tarjeta de red copia los paquetes a un anillo de descriptores, de donde HPCAP lee y copia a un buffer intermedio. Las aplicaciones leen directamente de ese buffer.}

\begin{figure}[hbtp]
\centering
\begin{tikzpicture}[scale = 0.7]
\begin{scope}[xshift = -6cm, yshift = -0.5cm]
\draw[fill = green!60!black, fill opacity = 0.3] (-0.1, 1.2) -- (0, 1.2) -- (0,-0.3) -- (0,0) -- (0.7,0) -- (0.7,-0.15) -- (1.7, -0.15) -- (1.7, 0) -- (1.8, 0) -- (1.8, -0.15) -- (2, -0.15) -- (2,0) -- (2.3, 0) -- (2.3, 1) -- (0,1) -- (0, 1.2) -- cycle ;
\end{scope}

\draw[->] (-3.6, 0) -- node[above, midway] {\small DMA} (-1.1, 0);

\fill[gray!20!white] (0,0) -- (0,1) arc (90:-30:1cm) -- cycle;
\foreach \x in {0, 30, ..., 360}
	\draw ({cos(\x)}, {sin(\x)}) -- ({-cos(\x)}, {-sin(\x)});
\draw (0,0) circle [radius = 1cm];
\draw[black, fill = white] (0,0) circle [radius = 0.5cm]; % Poor man's \clip for intersecting regions

\node[vnlin, label = {above:{\footnotesize tail}}] at (0,1.15) {};

\node[hnlin, label = {below right:{\footnotesize head}}, rotate = -30] at ({1.15 * cos(-30)}, {1.15 * sin(-30)}) {};

\node at (0, -1.8) {\small Anillo de descriptores};
\node at (8, -1.8) {\small Buffer de datos};
\node at (-4.85, -1.8) {\small NIC};

\fill[gray!20!white] (4, 0) rectangle (9, 0.5);
\draw (4, 0) rectangle (12,0.5);

\draw[thick] (5, 0) -- (5, 0.5);
\draw[thick] (5.5, 0) -- (5.5, 0.5);
\draw[thick] (7, 0) -- (7, 0.5);
\draw[thick] (9, 0) -- (9, 0.5);

\draw[->] ({1.05 * cos(75)}, {1.05 * sin(75)}) to [bend left] (4.2, 0.55);
\draw[->] ({1.05 * cos(45)}, {1.05 * sin(45)}) to [bend left] (5.2, 0.55);
\draw[->] ({1.05 * cos(15)}, {1.05 * sin(15)}) to [bend right] (5.7, -0.05);
\draw[->] ({1.05 * cos(-15)}, {1.05 * sin(-15)}) to [bend right] (7.2, -0.05);

\end{tikzpicture}

% \caption{Anillo de descriptores}
\end{figure}

\begin{itemize}
\item Un único hilo de copia al \textit{buffer} de datos.
\item Uso de \textit{padding} para poder almacenar por bloques. \note{El almacenamiento por bloques es más eficiente que paquete a paquete.}
\item Los descriptores se devuelven de manera ordenada a la tarjeta para su reutilización.
\end{itemize}

\end{frame}


\begin{frame}{Objetivos}

Modificación y mejora de HPCAP para la recepción a 40 Gbps:

\begin{itemize}
\item Recepción a 40 Gbps usando la menor cantidad de recursos posibles.
\item Marcado de tiempo preciso de los paquetes recibidos.
\item Desarrollo de métodos que permitan almacenar los datos recibidos.
\item Verificar la portabilidad de la arquitectura, usando tarjetas Intel y Mellanox.
\item Creación de un entorno de pruebas que permita verificar el rendimiento.
\end{itemize}

\end{frame}

\begin{frame}
\frametitle{Retos}
\note{Los retos a superar están por un lado en el desarrollo en sí, en la integración de un \textit{driver} que tiene que funcionar correctamente con las tarjetas de Intel y Mellanox e interactuar con el resto del \textit{kernel} Linux.

Pero además tiene que funcionar en un entorno restringido, donde estamos muy cercanos a los límites del hardware. Por ejemplo, un fallo del predictor de ramas puede suponer un cuarto del tiempo disponible para procesar un paquete de 64 bytes.

Además, la cantidad de datos generados es muy grande: si queremos almacenar todos los paquetes vamos a tener que reducirla de alguna forma.}

\begin{columns}
\begin{column}{0.5\textwidth}
\begin{itemize}
\item Las restricciones de tiempo están cerca de los límites de velocidad del procesador \footnotemark.
\item Tampoco hay mucho margen a nivel PCI: el ancho de banda máximo de PCIe 3.0 x8 es de unos 55 Gbps.
\item Se genera una gran cantidad de datos: 420 TB al día en un enlace 40 Gbps.
\end{itemize}
\end{column}
\begin{column}{0.5\textwidth}
\begin{figure}
\centering
\begin{tikzpicture}
\begin{axis}[
	width = 0.9\textwidth,
	horizontal mbarplot, xlabel = {\footnotesize Ciclos},
	symbolic y coords = {Cach\'e L3,Cach\'e L2,Cach\'e L1,Fallo \textit{branch}},
	ytick = data,
	x tick label style={font=\footnotesize},
    y tick label style={font=\footnotesize},
    xmax = 75,
    xtick = {0, 10, 20, 30, 40, 50, 57}
    ]
	\addplot plot coordinates {
		(4,Cach\'e L1)
		(12,Cach\'e L2)
		(44,Cach\'e L3)
		(16,Fallo \textit{branch})
	};
	\draw[mDarkBrown, dashed, thick] ({rel axis cs:{58/75},0}) -- ({rel axis cs:{58/75},1}) node[above, yshift = 0.05cm, midway, sloped, align = center, rotate = 180] {\footnotesize Tiempo entre  paquetes (64B)};
\end{axis}
\end{tikzpicture}
%\caption{Latencias en CPUs Intel Skylake a 3,4 GHz}
\end{figure}
\end{column}
\end{columns}

\footnotetext{\footnotesize Referencia: Intel 64 and IA-32 Optimization Reference Manual}
\end{frame}

\section{Desarrollo e implementación} \note{Empty}

\begin{frame}
\frametitle{Requisitos - Recepción 40G}

\note{En un primer momento podríamos decir que este problema ya está resuelto: ya existen colas concurrentes, ¿no? No, si pretendemos que vaya a 40 Gbps.}

Se necesita una arquitectura multihilo, pero con más restricciones de las habituales en sistemas concurrentes.

\begin{itemize}
\item El uso de semáforos, \textit{mutex} o \textit{spinlocks} para proteger regiones críticas es demasiado lento. \note[item]{En la medida de lo posible, tenemos que evitar que ningún hilo pare su ejecución mientras siga habiendo paquetes.}
\item Minimizar los puntos de sincronización para evitar fallos de caché y latencia por sincronización entre núcleos.
\item Minimizar el desorden de los paquetes recibidos.
\item Marcado de tiempo preciso.
\end{itemize}

Se pueden aprovechar las características de HPCAP para diseñar una arquitectura eficiente.
\end{frame}

\begin{frame}
\frametitle{Diseño (I)}

\begin{figure}
\centering
\begin{tikzpicture}[scale = 1.3]
\foreach \x in {0, 30, ..., 360}
	\draw ({cos(\x)}, {sin(\x)}) -- ({-cos(\x)}, {-sin(\x)});
\draw (0,0) circle [radius = 1cm];


\foreach[count = \i] \x/\c in {90/palette1, 0/palette2, -90/palette3, -180/palette4} {
	\draw[\c]
		({1.05 * cos (\x - 1)}, {1.05 * sin (\x - 1)})
		-- ({1.15 * cos (\x - 1)}, {1.15 * sin (\x - 1)})
		arc ({\x - 1}:{\x - 90 + 1}:1.15cm)
		-- ({1.05 * cos (\x - 90 + 1)}, {1.05 * sin (\x - 90 + 1)});

	\fill[opacity = 0.2, \c] (0,0) -- ({cos(\x)}, {sin(\x)})
		arc ({\x}:{\x - 90}:1cm) -- (0,0);

	\node[\c] at ({1.35 * cos (\x - 45)}, {1.35 * sin (\x - 45)}) {\i};

	\draw[very thick] (0,0) -- ({cos(\x)}, {sin(\x)});
}

\draw[black, fill = white] (0,0) circle [radius = 0.5cm]; % Poor man's \clip for intersecting regions
\node at (0,0) {Rx};

\fill[palette1, opacity = 0.2] (2, -0.25) rectangle (3, 0.25);
\fill[palette2, opacity = 0.2] (3, -0.25) rectangle (3.5, 0.25);
\fill[palette1, opacity = 0.2] (3.5, -0.25) rectangle (4, 0.25);

\foreach[count = \i] \x in {0,1,...,6} {
	\draw ({2 + \x / 2}, -0.25) rectangle ({2.5 + \x / 2}, 0.25);
}

\node at (4.5, -0.5) {\textit{Buffer} intermedio};

\draw (6, 0.25) -- (5.5, 0.25) -- (5.5, -0.25) -- (6, -0.25);
\node at (5.8, 0) {$\dotsb$};

\draw[palette1, dashed, ->] ({1.05 * cos (75)}, {1.05 * sin (75)}) to[bend left = 40] (2.25, 0.3);
\draw[palette1, dashed, ->] ({1.05 * cos (45)}, {1.05 * sin (45)}) to[bend left = 25] (2.75, 0.3);
\draw[palette2, dashed, ->] ({1.05 * cos (-15)}, {1.05 * sin (-15)}) to[bend right = 25] (3.25, -0.3);
\draw[palette1, dashed, ->] ({1.05 * cos (15)}, {1.05 * sin (15)}) to[bend left = 25] (3.75, 0.3);


\end{tikzpicture}

\end{figure}

\textbf{Anillo de recepción fijo}: se asigna un hilo a cada segmento del anillo de recepción. Así evitamos sincronizar la lectura, sólo se sincroniza devolución en el borde de cada segmento con una variable atómica. \note{La variable atómica marca si el segmento anterior ha acabado.}

Si está disponible, el marcado de tiempo hardware permitirá reordenar a posteriori los paquetes. \note{Si no tenemos marcado de tiempo hardware tenemos la ventaja de que tardamos menos en recuperar los paquetes, habría que medir bien si esto desordena menos que varios anillos.}

\end{frame}

\begin{frame}{Diseño (II)}

\begin{figure}
\centering

\begin{tikzpicture}[scale=1]
\fill[gray, opacity = 0.3] (0,-0.15) rectangle (1.5, 0.15);
\fill[gray, opacity = 0.3] (0,-1.15) rectangle (3.5, -0.85);
\fill[gray, opacity = 0.3] (0,-2.15) rectangle (7.5, -1.85);

\draw[dotted, thick, gray] (0, -0.15) -- (0, -2.15);
\draw[dotted, thick, gray] (2, -0.15) -- (2, -2.15);
\draw[dotted, thick, gray] (4, -0.85) -- (4, -2.15);
\draw[dotted, thick, gray] (6, -1.85) -- (6, -2.15);

\draw (0,-0.15) rectangle (2, 0.15);
\draw (0,-1.15) rectangle (4, -0.85);
\draw (0,-2.15) rectangle (8, -1.85);

\node[anchor = east] at (-0.2, 0) {\footnotesize Buffer (1 GB)};
\node[anchor = east] at (-0.2, -1) {\footnotesize Archivo (2 GB)};
\node[anchor = east] at (-0.2, -2) {\footnotesize Contador interno (4 GB)};

\draw (1.5, -0.2) node[below] (BL) {} -- (1.5, 0.2) node[above] (B) {};
\draw (3.5, -0.8) node[above] (A) {} -- (3.5, -1.2) node[below] (AL) {};
\draw (7.5, -1.8) -- (7.5, -2.2) node[above] (C) {};

\draw[shorten <= 0.2cm, shorten >= 0.01cm, ->, rounded corners] (C.north) |- (BL.south)  -- (BL.north);
\draw[shorten <= 0.2cm, shorten >= 0.01cm, ->, rounded corners] (C.north) |- (AL.south)  -- (AL.north);

\end{tikzpicture}

\end{figure}

\textbf{Buffers de tamaño prefijado}: se mantiene una variable atómica con la posición de escritura, dejamos que el \textit{overflow} de C haga el módulo implícitamente. \note[item]{Habría que reiniciar la posición a 0 cuando llegásemos al final del \textit{buffer}, lo que implicaría más sincronización. Pero si el tamaño del buffer es siempre potencia de 2, no tenemos que preocuparnos. Esto además nos permite controlar el padding al inicio y final del fichero.}
\end{frame}

\begin{frame}{Diseño (III)}

\begin{figure}
\centering

\begin{tikzpicture}
\draw[fill = palette1, fill opacity = 0.1, palette1] (0,0) -- (2,0) -- (2,3) -- (0,3);
\draw[fill = palette2, fill opacity = 0.1, palette2] (7,0) -- (5,0) -- (5,3) -- (7,3);

\node[palette1, anchor = east] at (2,3.2) {\scriptsize Kernel};
\node[palette2, anchor = west] at (5,3.2) {\scriptsize Espacio de usuario};

\foreach \x in {0,1,...,3}
	\node[rectangle, draw, black, fill = white] (RX\x) at (1.3, {2.5 - \x * 0.65}) {\scriptsize Rx \x};

\fill[gray, opacity = 0.3] (1.5, -1) rectangle (3, -0.7);
\draw (1.5, -1) rectangle (5.5, -0.7);

\draw[dotted, thick] (3, -0.58) -- (3, -1.12) node[below] (A) {};
\draw[thick] (3.8, -0.58) node[above] (B1) {} -- (3.8, -1.12) node[below ] (B) {};

\draw[thick, palette1, ->] (A.center) -- node[midway, below] {\scriptsize 1. Reserva} (B.center);

\draw[thick, palette1, shorten <= 0.1cm, ->] (RX3) to[bend left] node[midway, above, sloped] {\scriptsize 2. Copia} (3.3, -0.58);

\node[rectangle, draw, black, fill = white] (Lector) at (5.7, 0.4) {\scriptsize Lector};

\draw[thick, palette2, ->, shorten >= 0.1cm, shorten <= 0.1cm]  (2, 1.5) to[bend left] node[midway, sloped, fill = black!2!white] {\dots}  node[midway, above, yshift = 0.1cm] {\scriptsize 3. Nuevos datos} (Lector);

\draw[thick, palette2, shorten <= 0.1cm, ->]  (3.6, -0.58) to[bend left] node[midway, above, sloped] {\scriptsize 4. Lectura} (Lector);
\end{tikzpicture}

\end{figure}

\textbf{Lectores en espacio de usuario e hilos funcionando sin paradas}: no es necesario marcar cuándo se acaba la copia, ya que siempre acaba antes de que el lector sepa que hay nuevos datos.

\note{Con esto, ya podemos recibir a 40 Gbps. El problema es qué hacer con los datos.}

\end{frame}

\begin{frame}{Almacenamiento inteligente}

\begin{columns}
\begin{column}{0.45\textwidth}
\alert{Problema} 40 Gbps son muchos datos a procesar y almacenar ($\sim$400 TB al día).

Soluciones a nivel de hilos de recepción:
\begin{itemize}
\item \textbf{Filtrado}: Configurados en vivo por la aplicación lectora sin parar la captura en ningún momento.
\item \textbf{Almacenamiento selectivo}: Selección de los rangos de bytes que se guardan en el \textit{buffer}.
\end{itemize}
\end{column}
\begin{column}{0.55\textwidth}
\begin{figure}[btp]
\inputgnuplot[1]{gnuplot/caplen-effects}
\caption{Tasa efectiva limitando el tamaño de paquete}
\label{fig:Desarrollo:CaplenEffects}
\end{figure}
\end{column}
\end{columns}

\note{Empty}
\end{frame}

\begin{frame}{Sistemas multiprocesador}

\begin{figure}[tbp]
\centering
\footnotesize
\resizebox{0.7\textwidth}{!}{

\begin{tikzpicture}
\begin{scope}
\node[palette1, anchor = east] at (6, -2.3) {Nodo NUMA 0};
\draw[palette1, fill = palette1, fill opacity = 0.1] (0, -2) rectangle (6,2);
\draw[fill = white] (5.5, 1.2) rectangle (3.9, -1.2);
\draw (4.7, 1.2) -- (4.7, -1.2);
\draw (5.5, 0.4) -- (3.9, 0.4);
\draw (5.5, -0.4) -- (3.9, -0.4);

\foreach[count = \iy] \y in {0.8, 0, -0.8} {
	\foreach[count = \ix] \x in {4.3, 5.1} {
		\pgfmathsetmacro{\nodename}{2 * (\iy - 1) + \ix - 1}
		\node at (\x, \y) {\pgfmathprintnumber[int trunc]{\nodename}};
	}
}

\node at (4.7, 1.43) {CPU 0};

\draw[palette1, fill = palette1, fill opacity = 0.4] (0.2, 1.8) rectangle (1.8, -1.8);
\node[align = center] at (1, 0) {Memoria \\ local \\ nodo 0};

\fill[palette1, pattern = north east lines] (1.8, -0.6) rectangle (3.9, 0.6);
\node at (2.85, -0.8) {Bus local};

\node[draw, rectangle, minimum width = 2cm, minimum height = 1cm] (NIC) at (2.85, 3) {NIC};
\draw[very thick] (2.85, 0.6) -- (NIC);
\end{scope}

\begin{scope}[xshift = 14cm, xscale = -1]
\node[palette2, anchor = west] at (6, -2.3) {Nodo NUMA 1};
\draw[palette2, fill = palette2, fill opacity = 0.1] (0, -2) rectangle (6,2);
\draw[fill = white] (5.5, 1.2) rectangle (3.9, -1.2);
\draw (4.7, 1.2) -- (4.7, -1.2);
\draw (5.5, 0.4) -- (3.9, 0.4);
\draw (5.5, -0.4) -- (3.9, -0.4);

\foreach[count = \iy] \y in {0.8, 0, -0.8} {
	\foreach[count = \ix] \x in {5.1, 4.3} {
		\pgfmathsetmacro{\nodename}{2 * (\iy - 1) + \ix - 1}
		\node at (\x, \y) {\pgfmathprintnumber[int trunc]{\nodename}};
	}
}

\node at (4.7, 1.43) {CPU 1};

\draw[palette2, fill = palette2, fill opacity = 0.4] (0.2, 1.8) rectangle (1.8, -1.8);
\node[align = center] at (1, 0) {Memoria \\ local \\ nodo 0};

\fill[palette2, pattern = north west lines] (1.8, -0.6) rectangle (3.9, 0.6);
\node at (2.85, -0.8) {Bus local};


\node[draw, rectangle, minimum width = 2cm, minimum height = 1cm] (RAID) at (2.85, 3) {RAID};
\draw[very thick] (2.85, 0.6) -- (RAID);
\end{scope}

\fill[gray, opacity = 0.2] (5.5, 0.3) rectangle (8.5, -0.3);
\fill[pattern = horizontal lines, pattern color = gray] (5.5, 0.3) rectangle (8.5, -0.3);
\node at (7, -0.45) {\tiny Interconexi\'on};
\end{tikzpicture}
}
\caption{Arquitectura NUMA de los sistemas de pruebas.}
\label{fig:Validacion:NUMAArch}
\end{figure}

Para maximizar el rendimiento, se usan sistemas multiprocesador y se tiene en cuenta su arquitectura NUMA. Los hilos de recepción se ejecutan en el nodo con la NIC, los de almacenamiento en el nodo con el RAID.

\note{Empty}

\end{frame}

\section{Validación experimental} \note{Empty}

\begin{frame}{Entorno experimental}

\begin{columns}

\begin{column}{0.5\textwidth}
Servidores interconectados, uno genera tráfico y el otro recibe los datos con HPCAP. Aumentamos tasa de generación hasta encontrar el máximo soportado sin pérdidas.
\vspace{10pt}

Generador de tráfico: \textit{pktgen-DPDK}. No satura el enlace para tamaños pequeños y no controla la tasa de envío en tarjetas Mellanox. \note{Esto nos dará problemas para las pruebas de la Intel XL710.}
\end{column}

\begin{column}{0.5\textwidth}
\begin{figure}
\centering
\begin{tikzpicture}
\node[palette1, fill = palette1!30!white, text = black, rectangle, draw, minimum height = 2.5cm, minimum width = 1.6cm] (skadi) at (0,0) {\footnotesize Servidor 1};
\node[palette2, fill = palette2!30!white, text = black, rectangle, draw, minimum height = 2.5cm, minimum width = 1.6cm] (einsrepo) at (4,0) {\footnotesize Servidor 2};

\draw[<->] (skadi) -- node[midway, above] {\footnotesize 40 Gbps} (einsrepo);

\node[below = 0.1cm, align = center, scale = 0.55] at (skadi.south) {2x Intel Xeon E5-2620 \\ 2,4 GHz, 6 núcleos \\ 64 GB RAM \par};
\node[below = 0.1cm, align = center, scale = 0.55] at (einsrepo.south) {Intel Xeon E5-1620 \\ 3,7 GHz, 6 núcleos \\ 32 GB RAM \par};

\node[above = 0.1cm, scale = 0.6, align = center] at (skadi.north) {Mellanox MT27500 \\ ConnectX-3};
\node[above = 0.35cm, scale = 0.6, align = center] at (einsrepo.north) {Intel XL710};
\end{tikzpicture}

% \caption{Entorno de pruebas}
\end{figure}
\end{column}
\end{columns}

\end{frame}

\begin{frame}{Arquitectura básica + timestamp}

\begin{figure}[htbp]
\centering
\inputgnuplot[0.7]{gnuplot/simple-arch-max-rate}
\end{figure}

Se confirma que un único hilo no es suficiente. También se comprueba que el marcado por hardware o software da el mismo rendimiento. \note{Principalmente porque la marca que da la tarjeta hay que convertirla a una marca de tiempo válida.}

\end{frame}

\begin{frame}{Múltiples hilos}
\begin{figure}[hbtp]
\inputgnuplot[0.7]{gnuplot/multicore-max-rate}
%\caption[Capacidad del diseño de recepción a 40 Gbps]{Tráfico máximo soportado por el diseño de recepción a 40Gbps para 1, 2 y 4 hilos. El gráfico incluye la capacidad máxima teórica del enlace Ethernet.}
%\label{fig:Validacion:MulticoreMaxRate}
\end{figure}

Usando  múltiples hilos se mejora el rendimiento, pero alcanzamos un tope: Mellanox tiene un cuello de botella en la copia a memoria de los paquetes. \note{Empty}

\end{frame}

\begin{frame}{Desorden}
\begin{figure}[hbtp]
\inputgnuplot[0.7]{gnuplot/multicore-ordering}
%\caption[Desorden de paquetes inducido por el sistema de captura]{Porcentaje sobre el total de paquetes recibidos que se desordenan al pasar por el sistema de captura. Las barras de error muestran la variación de resultados al cambiar la velocidad y tamaño de paquete manteniendo el mismo \textit{ratio} de paquetes por segundo.}
\label{fig:Validacion:Ordering}
\end{figure}

No se introduce un desorden excesivo (máx. 30\% para tasas más altas). \note{Si tenemos marcado de tiempo hardware esto no es un problema.

30 Mpps sólo se alcanzan con paquetes pequeños, 150 bytes aprox.}
\end{frame}

\begin{frame}{Almacenamiento}
\begin{columns}
\begin{column}{0.5\textwidth}
\begin{figure}[hbtp]
\inputgnuplot{gnuplot/multicore-disk-store}
%\caption{Comparación entre rendimiento descartando paquetes y guardándolos en un sistema de archivos en RAM.}
\label{fig:Validacion:DiskStore}
\end{figure}
\end{column}
\begin{column}{0.5\textwidth}
\begin{figure}[hbtp]
\inputgnuplot{gnuplot/caplen-effects-experimental}
%\caption[Tasa máxima experimental limitando el tamaño de paquete]{Tasa máxima de recepción guardando los datos en RAM y limitando el tamaño de paquete.}
\label{fig:Validacion:CaplenEffects}
\end{figure}
\end{column}
\end{columns}
\vspace{10pt}
Las pruebas se realizan con un buffer de 2 GB, que es el valor en el que ya se consigue la tasa máxima.\note{Comentar lo del parche aceptado.}

El guardado de paquetes reduce la tasa de recepción. Si se reducen los datos a guardar se consigue llegar a la tasa máxima del sistema.
\end{frame}

\begin{frame}{Resultados Intel}

\begin{figure}[hbtp]
\inputgnuplot[0.7]{gnuplot/intel-hpcapdd}
%\caption[Tasa de recepción para la tarjeta Intel XL710]{Tasa de recepción de tráfico con guardado en RAM de los datos para la tarjeta Intel XL710.}
\label{fig:Validacion:IntelHpcapdd}
\end{figure}

Problemas con el generador: no satura enlace y no permite controlar la tasa de envío. Sólo se obtienen resultados significativos para la prueba de guardado en RAM.

Aun así, se comprueba que la arquitectura diseñada mejora considerablemente el rendimiento obtenido.
\note{Como no controlamos tasa de envío, el porcentaje de pérdidas no es útil porque varía según la tasa a la que enviemos.}
\end{frame}

\section{Conclusiones} \note{Empty}

\begin{frame}{Resultados}
\begin{itemize}
\item Diseño válido para monitorización a 40 Gbps a partir de 600 bytes de tamaño de paquete.
\item Arquitectura y código portables entre \textit{driver} Intel y Mellanox.
\item El almacenamiento selectivo y el filtrado son capaces de mejorar el rendimiento de la captura y reducir la cantidad de datos a guardar.
\item El marcado de tiempo por hardware permite tolerar un desorden mayor en la recepción sin afectar al rendimiento.
\item Durante el desarrollo se ha detectado un \textit{bug} en el \textit{kernel}, que se ha corregido y ha sido aceptado en Linux.
\end{itemize}

\note{Hemos conseguido los objetivos básicos que nos planteábamos}
\end{frame}

\begin{frame}{Trabajo futuro}
\begin{itemize}
\item Incorporación de filtros ASCII o BPF.
\item Limitaciones de tamaño dinámicas en función de la tasa de recepción.
\item Mejora del entorno de pruebas: generación a 40 Gbps y soporte de flujos TCP.
\end{itemize}

\note{Aunque todavía queda bastantes cosas por hacer y mejorar.}
\end{frame}

\begin{frame}[standout]
¿Preguntas?

\note{Fin.}
\end{frame}

\end{document}
